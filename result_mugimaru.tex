\subsubsection{本走行の結果}
むぎまるチームは,市役所南東の交差点で
直進するところを右折し,リタイアした.
これは,開発したsimple\_waypointfollowerが,
間違った場所を目的地として配信していたためである.
原因は現在究明中であるが,
つくばチャレンジの環境以外では,見られていない.

\subsubsection{本走行・実験走行で見つかった課題と失敗}
つくば市街での走行時に見つかった課題として,
走り出すまでの時間が長いという点がある.
価値反復のノードが目的地の座標を受け取ってから
目的地に走り出すまで,価値反復の計算量が多いために
時間がかかる.
具体的に計測したわけでないが,30mほど先の目的地までに10秒から20秒
ほどその場で旋回したり目的地とは反対方向に向かったりした.
この問題を解決する方法としてA*を併用する方法\cite{中村2024, 中村2025}がある.
具体的には, ロボットにゴールが与えられたとき,そこまでの経路をA*探索で先に解き,
その周辺の暫定的な方策を計算することで,ロボットの走り出しを早めるというものである.
今年度の実験走行と本走行では,試すことはできなかったが,
来年度はこの手法を用いて走行したいと考えている.

実験走行での失敗として,
バッテリーを増量したら
ギアボックス内のギアが削れ,動作不能になった.
元々Raspberry Pi Catは小型で軽量であるため,
付属するギアボックスの耐荷重を超えたものと考えれる.
初歩的な失敗では,オドメトリに使用するimuの電源をつけ忘れたり,
ロボットの諸元(ホイールの直径とトレッド)を間違えたりした.

