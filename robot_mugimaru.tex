\subsubsection{概要}
むぎまるチームは1章で述べた目標のために, 
価値反復ROSパッケージ(value\_iteration2\cite{value_iteration2})\cite{ueda2023JRM}と,
ともに使用するナビゲーション用のパッケージを開発した.
価値反復ROSパッケージでは,自己位置推定が十全に行えていることを前提としている.
そのため,自己位置推定には,昨年度のつくばチャレンジで動いた実績のある
きなこチームのものをそのまま用いている.
% パッケージであるemcl2\_ros2\cite{emcl2_ros2}を使用した. 

\subsubsection{システム構成}
むぎまるチームでは,Raspberry Piでオドメトリと車体の制御,
miniPCで自己位置推定と経路計画を行っている.
ロボットに搭載した計算機, センサ, アクチュエータの接続の関係及び
計算機で実行するROS 2ノードの概要を表したものを
図\ref{fig:mugimaru_system}に示す. 

Raspberry Piには, 車体制御のために, IMUとエンコーダ, 車輪駆動用のモータを繋ぎ,
LANケーブルを用いてminiPCと接続している.
実行するノードとしては, IMU用のROS 2ドライバノード(rt\_usb\_9axisimu\_driver)と, 
オドメトリの出力とモータの制御を実施するノード(raspimouse)がある. 
後者のノードでは, エンコーダの値と
前者のノードで配信されたIMUの情報からロボットのオドメトリを計算し, 
トピックとして配信する. 
モータの制御は, 外部のノードからトピックとして受信した
ロボットの速度指令をもとに実行する. 

miniPCではRaspberry Piからの
オドメトリと接続した3D LiDARの情報から自己位置推定を行い,
その結果を使って経路計画を行う.
Raspberry Pi と3D LiDARは,LANケーブルを介して接続するが,
miniPCポートが足りないため,スイッチを使用している.
実行されるノードとしては, マップと3D LiDARに関係するもの: 
\begin{itemize}
	\item livox\_lidar\_publisher: Livox用のROS 2ドライバノード
	% \item pointcloud\_to\_laserscan: 3D LiDARのスキャンデータを2次元に圧縮してトピックとして配信するノード
	\item map\_manager: 複数の2次元マップと高さ範囲のパラメータを管理
    \begin{itemize}
      \item ロボットが指定した領域に進入すると, その領域に最適化された2次元マップと高さパラメータを提供
    \end{itemize}
    % \item pointcloud\_to\_dual\_scan: 3D LiDARのスキャンデータから以下の2種類の2次元スキャンデータを生成
    \item pointcloud\_to\_laserscan: 3D LiDARのスキャンデータから以下の2種類の2次元スキャンデータを生成
        \begin{itemize}
          \item 障害物回避用のスキャンデータ
          \item 自己位置推定用のスキャンデータ(map\_managerの指定する高さ範囲に基づく)
        \end{itemize}
\end{itemize}
と, 自己位置推定とナビゲーションに関係するもの: 
\begin{itemize}
	\item emcl2\_ros2: 自己位置推定ノード(図中では「emcl2」と表記)
	\item 船井さんのやーつ
	\item value\_iteration2: 価値反復のノード
	\item simple\_waypointfollower: ゴールポーズを提供するノード
\end{itemize}
がある. 
map\_managerは, 事前に作成した複数の2次元占有格子地図を保持している. 
これらの地図は, 3D LiDARとIMUを統合したSLAM技術であるGLIM\cite{glim}\cite{glim_github}を用いて作成した3次元マップから, 
必要な高さ領域を切り出して変換したものである. 
切り出しにはpointcloud2pgm\_slicer\cite{pc2pgm_github}を用いた.

% 走行する環境の地図は,GLIMで作成したものを
% 高さ方向で切り出したものを使用した.
% 切り出しにはpointcloud2pgm\_slicerを用いた.

%書き換える
\begin{figure}[h]
  \begin{center}
    \includegraphics[width=1.0\linewidth]{figs/mugimaru_system.eps}
    \caption{むぎまるチームのシステム構成}
    \label{fig:mugimaru_system}
  \end{center}
\end{figure}

\subsubsection{価値反復ROSパッケージ}
価値反復ROSパッケージは,価値反復アルゴリズムを
移動ロボットの経路計画へ適用したものである.
価値反復は,環境内のすべての位置と角度に対して,
目的地までの時間とその位置の移動時間への悪影響を合計したコスト
と呼ばれる数値を計算するアルゴリズムである.
このコストが小さくなるように行動することで,目的地へたどり着くことができる.
このコストと行動は,計算が収束すると最適なものとなる保証がある.

value\_iteration2では,
ナビゲーション用の地図にない障害物の位置も,
地図に上書きして最適方策を求めることができる.
これにより, 後から出現した障害物の位置を考慮に入れた
最適方策を計算できる.
最適方策は障害物を避けるように少しずつ更新されるため,
経路の再計算の遅延や, 再計算を繰り返すことによるチャタリング,
他のプランナーや動作との競合で
ロボットが動かなくなることはない.
また, 「再計算」という概念がないにも関わらず,
障害物が未知を塞いだ場合にう回する方策を算出できる.

ただし, 変わった障害物の位置に対して常に最適方策を
計算し続けるため, CPUを多く使い電力を大きく消費する.
具体的には,2時間から3時間ほど計算のみ(モータやセンサの駆動は別のバッテリー)させると,
62500mAh(200Wh)のバッテリーを消費しきる程度である.

詳細な価値反復アルゴリズムの説明は,
文献\cite{上田2019, 上田詳解}にある.
