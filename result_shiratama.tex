\subsubsection{本走行の結果}
しらたまチームはつくば市役所外周を走行中, 
曲がり角にて他チームのロボット(以下,相手機体)と接触し,
141mで終了となった.
その時の状況を以下で述べる.  

%その時の状況
 つくば市役所外周の曲がり角付近において,コース外側を走行していたしらたまチームのロボットが左折動作を試みた際,コース内側を走行していた相手機体と接触の危険が生じたため,回避行動を実施した.
 その後,しらたまチームのロボットが再度左折動作を行ったところ,相手機体の走行進路上に進入し,しらたまチームのロボットの側方部と相手機体の前方部が接触した. 
 当該接触により,しらたまチームのロボットは走行継続が困難であると判断し,本走行を終了した. 
\subsubsection{本走行・実験走行・学内での実験で見つかった課題}
\paragraph{未知障害物によりMCLが破綻する問題}
 しらたまチームのロボットは,2D LiDARのみをセンサとして用い,「Monte Carlo localization (MCL)」に基づく自律走行を実施していた. その結果,実験走行中に自律走行の継続が困難となる事例が複数確認された.

 具体的には,他チームのロボットや一般の通行人とのすれ違いが頻発する状況において,自己位置推定が不安定となり,コース外の位置を走行してしまう,あるいはその場で停止してしまうといった挙動が観測された. また,本走行当日には,市役所庁舎裏側にて猫の譲渡会が開催されており,当該会場前を通過する際にも同様に自己位置推定が不安定になり,ロボットが停止する事例が確認された.

 これらの事例は,文献\cite{ikebe2023}において指摘されている未知障害物の存在によるMCLの破綻に該当すると考えられる. 一般の通行人や他チームのロボット,ならびに猫の譲渡会会場への多数の来場者は,事前に作成された地図には含まれない未知障害物であり,これらが大きく影響した結果,自己位置推定の破綻を招き,走行継続が困難となったものと考えられる. 特に,本システムでは2D LiDARのみを用いたMCLによる自己位置推定を採用していたため,文献\cite{ikebe2023}で述べられているように未知障害物に対するなんらかの対策を,MCL側で講じる必要があると考えられる. 
\paragraph{緊急停止ボタンの接続不備により走行できなかった問題} しらたまチームは学内での実験走行中に, ロボットが走行できない問題に直面した. 緊急停止ボタンの接続不備により, 
モータに電力が供給されていなかったことが原因だと判明した.  
ソフトウェア側の問題だと考えて調査していたが, 
最終的にハードウェア側の原因だと特定するまで長い時間がかかった. 
