\subsubsection{本走行の結果}
しらたまチームはつくば市役所外周を走行中, 
曲がり角にて法政大学 自律ロボット実験室Aの機体(Orange papa 2025)と接触し,
141mで終了となった.
その時の状況を以下で述べる.  

%その時の状況
つくば市役所外周の曲がり角付近において,コース外側を走行していたしらたまチームのロボットが左折動作を試みた際,コース内側を走行していた法政大学自律ロボット実験室Aのロボットと接触の危険が生じたため,回避行動を実施した. 
その後,しらたまチームのロボットが再度左折動作を行ったところ, 法政大学自律ロボット実験室Aのロボットの走行進路上に進入し, しらたまチームのロボットの側方部と法政大学自律ロボット実験室Aのロボットの前方部が接触した. 
当該接触により, しらたまチームのロボットは走行継続が困難であると判断し, 本走行を終了した. 
\subsubsection{本走行・実験走行・学内での実験で見つかった課題}
\paragraph{緊急停止ボタンの接続不備により走行できなかった問題}
しらたまチームは、学内での実験走行中に、ロボットが走行できない問題に直面した。
原因は、緊急停止ボタンの接続不備により、
モータに電力が供給されていなかったことだと判明した。 
ソフトウェア側の問題だと考え、調査していたが、
最終的にハードウェア側の原因だと特定するまで長い時間がかかった. 
