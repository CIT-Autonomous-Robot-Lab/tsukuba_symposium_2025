\subsubsection{本走行の結果}
きなこチームは本走行において, $588$mの走行を達成した.
駐車場の出入り口を横切る際に一時停止をする必要があるが,
人為的なミスにより停止せずに通過してしまい,
そこで走行終了となった.
また, その後も自律走行を継続し, $852$mまで走行した.
しかし, 信号機の前で停止をしなかったため, そこでリタイアとなった.
この原因は特定できていない.


\subsubsection{実験走行}
実験走行では, ノートPCを使わず, Raspberry Pi 4にすべての処理を行わせて走行した.
その結果, スタート地点から最長$150$m走行することができた.
実験走行の段階では圧縮前の地図に歪みがあったため, これ以上の結果は得られなかった.
しかし, 圧縮前の地図容量($30.9$GB)よりも小さいメモリ容量($4$GB)の計算機を用いても,
走行可能であることが確認できた.

\subsubsection{得られた知見}
自己位置推定では,
点群の高さ切り替えを行うことで人混みや未知障害の影響を受けにくくなり,
地図の歪みを解消してからは自己位置推定が破綻することはなくなった.

ナビゲーションでは,
FMMに基づくベクトル場を用いることで,
対向ロボット・動的障害物に対して早めに回避を開始でき,
衝突の心配がなく距離を保ちながら走行することができた.
さらに,
障害物マスクを反映したポテンシャル場の勾配により開けた場所を選択しやすいためか,
スタックは確認されなかった.
