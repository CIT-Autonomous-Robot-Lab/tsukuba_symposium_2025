\subsubsection{本走行の結果}
きなこチームは本走行において, 588mの走行を達成した.
駐車場の出入り口を横切る際に一時停止をする必要があるが,
人為的なミスによりこれを行わず,
走行終了となった.
その後も自律移動を継続し, 852mまで走行した.
信号機の前で停止をしなかったため, そこでリタイアとなった.
この原因は特定できていない.


\subsubsection{実験走行}
実験走行では, ノートPCを使わず, Raspberry Pi 4ですべての処理を行わせて走行した.
スタート地点から最長150m走行することができた.
実験走行の段階では圧縮前の地図に歪みがあったため, これ以上の結果は得られなかった.
しかし, 圧縮前の地図容量よりも小さいメモリ容量の計算機を用いても,
走行可能であることが確認できた.

また, 
自己位置推定で使う点群の高さ切り替えを行ったことで人混みや, 
未知障害の影響を受けにくく, 
地図の歪みを無くしてからは自己位置推定が破綻することはなくなった. 

また, FMMを採用したベクトル場によるナビゲーションで
対向ロボット・動的障害物に対して早めに回避することで, 衝突の心配がなく, 距離を取りながら走行することができた. 
また, 〜により, ひらけた場所に向かうため, 
スタックすることも無かった. 

% モード切り替えに伴う判断の遅延やスタックが解消され,
% 一貫性のある滑らかな挙動を実現した.